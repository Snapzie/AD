\section{Task 2}

Bevis korrektheden af din algoritme i task 1:\\

Vi beviser korrektheden for vores algoritme pr. induktion, hvor vi som basistilfælde betragter et træ af størrelse 1. For dette tilfælde ser vi, at vi hvis $k$ ikke er gyldigt, da returneres NIL, som ønskes. I tilfælde af et gyldigt $k$, da kan kun nøglen for træet eneste element returneres, hvilket ses at være tilfældet i linje 4-5.\\
Vores induktive antagelse er nu, at algoritmen gælder for træer af størrelse $1$ til $s$. Træet med størrelse $s$ vil i følgende kaldes for $S$.
Vi viser nu, at algoritmen dermed også vil virke for et træ $T$ af størrelse $s+1$.
Først og fremmest ser vi, at hvis $k$ er ugyldigt, så vil der returneres NIL, som ønskes.\\
Hvis vi ender i tilfældet, hvor vi betragter et træ med kun et element, da returnerer vi nøglen for netop dette element.\\
Hvis det for træet $S$ gælder, at det k'te element er i roden, så vil dette gælde for træet $T$, hvis det tilføjede element mellem træet $S$ og $T$ er i højre deltræ. Hvis det tilsatte element er i venstre deltræ, så vil det $k'te$ mindste element være i venstre deltræ, hvorfor linje 8-9 vil eksekveres og vi rekursivt kalder algoritmen på venstre deltræ. Vi husker på vores antagelse om, at algoritmen virker for træer af størrelse mellem $1$ og $s$. Dvs, at algoritmen vil virke på dette venstre deltræ, da denne nødvendigvis må have en størrelse mellem $1$ og $s$. \\
Hvis det for træet $S$ gælder, at det $k'te$ mindste element ligger i højre deltræ og vi går igennem else-klausulen, så vil det $k'te$ mindste element for træet $T$ enten ligge i højre deltræ eller være roden. I tilfælde af, at elementet ligger i højre deltræ, så formindskes problemstillingen til en, som ligger i størrelsen mellem $1$ og $s$, som antages at være korrekt. Desforuden sker vores rekursive kald i højre træ på vores opdaterede parameter, som ses i linje 11, for at reflektere, at vi nu essentielt set har 'bortkastet' venstre deltræ så vi altså opdatere vores værdi af $k$.
Hvis elementet er roden, da vil nøglen for denne returneres.\\
Vi ser altså, at vores indsuktions-skridt altid gælder, idet det gælder for størrelsen $1$ til $s$, hvorfor vores algoritme pr. induktion er bevist korrekt. 

