\section{Task 2}

Først viser vi, at problemstillingen indeholder 'The greedy choice property', hvilket vi gør ved at vise, at der findes en optimal løsning til problemstillingen, som indeholder det grådige valg.\\
Først definerer vi det grådige valg \text{g} for vores problem \text{P}:\\
Hvis en bar \text{b} mangler/er i overskud af \text{n} øl for at have det nøjagtige antal øl, da vil det grådige valg være at sende eller anmode om så mange øl, at der er nøjagtigt det rigtige antal øl i \text{b}, og at disse anmodes fra/sendes til den nærmeste bar for at minimere spild. Mængden af øl, der sendes/anmodes om, vil i følgende (og i algoritmen selv) kaldes for 'potentialet'.\\
Vi viser nu, at \text{g} ligger i en optimal løsning til \text{P} (Hvor en 'optimal' løsning blot er en bekræftelse af, hvorvidt der kan være $\hat{b}$ øl i hver bar):\\
Vi antager, at vi har en optimal løsning til \text{P} som IKKE indeholder \text{g}, men i stedet et andet valg \text{u}. Vi kan så erstatte dette valg \text{u} med \text{g}, da \text{g} strengt taget er bedre end (eller lige så godt) som \text{u}. Det betyder, at hvis den ikke-grådige løsning bekræfter, at der kan være $\hat{b}$ øl i hver bar, så vil den grådige løsning ligeså bekræfte dette. Hvis den ikke-grådige løsning afkræfter, at der kan være $\hat{b}$ øl i hver bar, da vil den grådige løsning enten afkræfte eller bekræfte, alt efter om hvorvidt det mindskede spild gør det muligt at klemme $\hat{b}$ øl ind i hver bar. Dvs. at vores løsning enten forbliver den samme eller bedre ved at træffe det grådige valg, hvorfor vi har 'The greedy choice property'.\\
Nu viser vi optimal delstruktur for vores problem \text{P}, hvilket vi gør ved at vise, hvis et grådigt valg \text{g} ligger i en optimal løsning til \text{P}, da vil denne optimale løsning bestå af \text{g} og en optimal løsning til det mindre, resterende problem \text{P'}.\\
Vi antager, at vi har truffet det grådige valg \text{g}.
Dvs. at den bar nu er ude at ligningen, og vi har en mængde af barer på én mindre. Vi kan derfor starte forfra på denne mængde af barer og checke fra en anden af og checke for hver bar, hvorvidt den mangler øl eller ej og sende potentialet videre derefter. Dvs., at dette nye delproblem \text{P'} også vil have 'The greedy choice property', hvorfor det også vil have en optimal løsning indeholdende det grådige valg. Vi ser altså for hvert delproblem, vil det have en optimal løsning indeholdende det grådige valg, hvorfor det altså også for vores originale problemstilling \text{P} vil gælde, at hvis \text{g} ligger i en optimal løsning for \text{P}, da vil denne bestå af \text{g} og en optimal løsning til \text{P'}, hvorfor vi har optimal delstruktur.

Nu vi har vist både 'The greedy choice property' og optimal delstruktur, da ved vi, at en optimal løsning til \text{P} indeholder en række af udelukkende grådige valg, hvilket vil blive brugt til at bevise korrektheden af vores algoritme.\\
I while-lykken ser vi, at vi udregner det grådige valg som "potentialet", der sendes videre til næste bar for at angive, om der sendes eller anmodes om øl (hvortil det huskes, at der skal fratrækkes 2 gange distancen for at angive, at der ved at sende øl mistes øl i processen). Hvis at det koster mere øl at sende en bars overskydende øl, end det har i overskud, sættes potentialet til 0 for at indikere, at det grådige valg er IKKE at sende øl videre.
Vi bemærker altså, at vi i while-løkken for hver bar træffer det grådige valg, som vi pr. "The greedy choice property" ved er indeholdt i en optimal løsning til \text{P}. Derefter gør vi det samme for næste bar, som reducerer problemstillingen til et nyt delproblem \text{P'}, som vi pr. optimal delstruktur ved også har en løsning indeholdende det grådige valg \text{g}. Dette gør vi for alle barer ved at iterere over dem enkeltvis. Dermed træffer vi altså en sekvens af udelukkende grådige valg, hvorfor vi finder en optimal løsning.
Efter while-løkken checker vi så, om sidste bar har et potentiale større end nul\todo{B\_bar?}, kun i hvilket tilfælde, det er muligt at udstyre hver en bar med det ønskede antal øl. 
