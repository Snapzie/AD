\section{Task 3}
I følgende skrives pseudokode til at bestemme den maksimale antal øl $\hat{b}$, som enhver bar kan have. Koden tager udgangspunkt i vores algoritme fra task 1 (I følgende kaldet Alg), som kører i $O(n)$ tid.

Først indser vi, at det absolut maksimale antal øl en bar kan have er gennemsnittet af samtlige øl fordelt over alle barer. Vi bestemmer dette gennemsnit $s$, som vil være af størrelsesordenen $O(B)$.
Dernæst foretages der binær søgning på intervallet $0$ til $s$, hvor at algoritmen fra task 1 køres på hver værdi. Pseudokode er:\\

\begin{lstlisting}
  AlgBin(b_bar, B, P):
      s = 0
      for i = 1 to n do:
          s+=B[i]
      s = s/n
      L = 0
      R = s
      while (L <= R) do:
          s = floor(L+R/2)
          if (Alg(s, B, P)):
            L = s+1
          else:
            R = s-1
      return s
\end{lstlisting}

Denne algoritme må nødvendigvis være korrekt, da den i hver iteration af while-løkken leder i et interval mellem min og max, indtil 
dette interval er blevet indsnærpet til netop det ene element, der i så fald returneres. 
Køretiden må således være $O(nlog(B))$, da der først summeres i $n$ tid, og dernæst køres algoritmen fra task 1 maksimalt $log_2(B)$ gange, 
hvorfor køretiden samlet bliver $O(nlog(B))$.

