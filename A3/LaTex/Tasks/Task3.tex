\section{Task 3}
Vores algoritme tager udgangspunkt i den grådige algoritme fra Task 1 som er vist at køre i $O(n)$ tid.
Først summerer vi det samlede antal af øl op til en sum s. Dette gøres ved at gå hver bar igennem enkeltvis og tager derfor O(n) tid.
Vi bemærker, at når B angiver antallet af øl, som baren med flest øl har, da vil $s \leq nB$,\todo{\\leq?} når der er n barer.
Dvs. s = O(nB).
Vi definerer nu den maksimale mængde øl $\hat{b}$ som $\hat{b} = \frac{s}{n}$, hvilket vi gør, da der maksimalt kan være $\frac{s}{n}$ øl i hver bar, hvis de er perfekt distribueret uden at nogen øl er gået tabt i transport. Vi kører nu algoritmen fra task 1 på dette tal $\hat{b}$ og checker, om det kan lade sig gøre. Hvis ikke, så forsøger vi med én mindre øl i hver bar, hvilket svarer til, at vi fratrækker n fra s og bestemmer $\hat{b}$, hvorefter vi endnu en gang kører algoritmen fra Task 1 med vores nye værdi for $\hat{b}$. Denne procedurer fortsættes indtil vi finder en værdi for $\hat{b}$, for hvor algoritmen fra Task 1 returnerer True.
Vi bemærker altså, at vi kører algoritmen fra Task 1, som kører O(n) tid, maksimalt $\frac{s}{n}$ gange. Og da s = O(nB), da vil $\frac{s}{n} = \frac{O(nB)}{n} = O(nlog_n(B))$.
Samlet får vi altså en køretid på $O(n + nlog_n(B)) = O(nlog(B))$.
