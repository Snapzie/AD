\section{Task 1}
\begin{lstlisting}
	Alg(b_bar, B, P):
		i = 0
		while(i != len(P) - 1):
			diff = B[i] - b_bar
			dist = P[i+1] - P[i]
			pot = diff - (2 * dist)
			if(diff > 0 & pot < 0):
				pot = 0
			B[i+1] += pot
			i += 1
		if(B[len(P) - 1] > 0):
			return True
		else:
			return False
\end{lstlisting}
Pseudokoden for algoritmen antager at listerne 'P'og 'B' er defineret som henholdsvis barenes positioner og barenes antal af øl.
Vi bemærker, at vi starter ved første bar og så for hver bar checker, hvilket potentiale, vi skal sende videre til næste bar. Dvs., at vi kun checker hver bar én gang, og hvert check er i form af en konstant-tids udregning, hvorfor vores algoritme kører O(n).
