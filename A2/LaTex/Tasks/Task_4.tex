\section{Task 4}
\subsection{Bevis for korrektheden af algoritmen}

Da algoritmen udregner selv samme som i Task 1, men denne gang blot gemmer resultater undervejs, opnår denne dynamiske implementering samme resultat som algoritmen i Task 1, hvorfor den nødvendigvis må være korrekt. 

\subsection{Bevis for køretiden af algoritmen}

Vi har et array på størrelse $Cn$, som undervejs udfyldes.
En plads kan maksimalt tage $C$ kald at udfylde, idet hver øl antages at koste minimum $1$ således, at der maksimalt kan købes $C$ øl i én mængde. 
Når først hver plads er blevet beregnet, da udføres fremtidige kald til den pågøldende plads i konstant tid. Idet algoritmen terminerer, foretages der et sidste kald, som ligeså udføres i konstant tid, hvorfor køretiden bliver $O(Cn+\Theta(1)) = O(Cn)$.

\subsection{Hukommelsesforbrug af algoritmen}
Algoritmen gør brug af et array af størrelse $Cn$, dvs.
at algoritmen har et hukommelsesforbrug på $O(Cn)$.

