\section{Task 2}

Vi bemærker, at når vi skal bestemme antallet af mængder på $i$ øl,
så for hver øl træffer vi en beslutning om hvorvidt en øl skal købes eller ej.
Dette giver to scenarier: Et hvor øllen købes og ens penge formindskes med prisen på den pågældende øl, og et hvor man ikke køber øllen og ens mængde penge forbliver det samme. I begge tilfælde opstår der dermed to mindre delproblemer,
hvor man skal finde antallet af kombinationer på en mængde øl, som er én mindre end før. Derfor består løsningen på vores problemstilling af løsninger på mindre delproblemer.

Vores rekursive formel tester altså samtlige tilfælde, for hvert delproblem, som vores originale problemstilling består af, og tæller kun op, hvis den finder en egentlig løsning, hvorfor den nødvendigvis må være korrekt.

Vores overordnede problemstilling består af overlappende delproblemer hvis der er en eller flere øl som kan fjernes fra vores sæt og give det samme sæt som hvis andre øl blev fjernet i stedet. Hvis der på flere måder kan dannes det samme sub-set af øl, så vil vi møde det præcis samme delproblem og vi kan derved benytte potentialet af dynamisk programmering.

