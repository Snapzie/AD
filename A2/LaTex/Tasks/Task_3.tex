\section{Task 3}
Vi bliver bedt om at omsrive vores rekursive formel til pseudokode.
\begin{lstlisting}
	N(C, i)
		if (C == 0)
			return 1
		else if (i == 0 || c < 0)
			return 0
		else if (A[C, i] != -1)
			return A[C, i]
		else
			A[C, i] = N(C - P[i], i - 1) + N(C, i - 1)
			return A[C, i]
\end{lstlisting}
'C' angiver vores penge mens 'i' indexerer i vores array 'P' som er en liste af priser på øl. Prisen for den i'ende øl er altså P[i]. Array 'A' er et pris array som indeholder priser på de delproblemer som allerede er løst.

