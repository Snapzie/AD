\section{Task 1}

\subsection{1}

$p(n) = 8p(\frac{n}{2}) + n^2$\\
Vi bruger sætning 1 side 94 og får, at $a = 8$, $b = 2$ og $f(n) = n^2$.
Dvs. vi har:
$$n^{log_b(a)} = n^{log_2(8)} = \Theta(n^{log_2(8)}) = \Theta(n^3)$$
Fordi $n^2 = O(n^{3-\varepsilon})$ hvor $\varepsilon \leq 1$ for alle $n \geq 0$, gælder det pr. sætning 1, at $p(n) = \Theta(n^3)$.

\subsection{2}

$p(n) = 8p(\frac{n}{4}) + n^3$\\
Vi bruger sætning 3 side 94 og får, at $a = 8$, $b = 4$ og $f(n) = n^3$.
Dvs. vi har:
$$n^3 = \Omega(n^{log_4(8)+\varepsilon}) = \Omega(n^{\frac{3}{2}})$$
hvilket gælder for $\varepsilon \leq \frac{3}{2}$ og for alle $n \geq 0$.
Derudover skal det gælde, at $8(\frac{n}{4})^3 \leq cn^3$ for $c < 1$ for alle $n \geq n_0$. Vi omskriver:
$$8(\frac{n}{4})^3 = 8 \cdot \frac{1}{4} \cdot n^3 = 2n^3$$
Dvs. for $c \geq 2$, da vil $2n^3 \leq cn^3$.
Dvs. at $p(n) = \Theta(n^3)$.

\subsection{3}

$p(n) = 10p(\frac{n}{9}) + nlog_2(n)$\\
Vi bruger sætning 1 side 94 og får, at $a = 10, b = 9$ og $f(n) = nlog_2(n)$.
Dvs. vi har:
$$nlog_2(n) = O(n^{log_9(10)-\varepsilon}) = O(n^{1.048 - \varepsilon})$$
hvilket gælder for alle $\varepsilon \leq 0.48$ og for alle $n \geq 0$.
Dette afledes af den generelle regel, at $log_a(x) = O(x^b)$ for $a > 1$ og $b > 0$.
Fordi $nlog_2(n) = O(n^{1.048 - \varepsilon})$ for $\varepsilon \leq 0.48$, gælder det pr. sætning 1, at $p(n) = \Theta(n^{1.048})$.

